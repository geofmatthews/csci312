\documentclass{beamer}
\usetheme{Singapore}
\usepackage{changepage}

%\usepackage{pstricks,pst-node,pst-tree}
\usepackage{amssymb,latexsym}
\usepackage{tikz}
\usepackage{graphicx}
\usepackage{fancyvrb}
\usepackage{hyperref}
\usepackage{fancybox}
\usepackage[listings]{tcolorbox}

\definecolor{codegreen}{rgb}{0,0.6,0}
\definecolor{codegray}{rgb}{0.5,0.5,0.5}
\definecolor{codepurple}{rgb}{0.58,0,0.82}
\definecolor{backcolour}{rgb}{0.95,0.95,0.92}

\lstdefinestyle{mystyle}{
    language=Lisp,
    backgroundcolor=\color{backcolour},   
    commentstyle=\color{codegreen},
    keywordstyle=\color{magenta},
    numberstyle=\tiny\color{codegray},
    stringstyle=\color{codepurple},
    basicstyle=\ttfamily\footnotesize,
    breakatwhitespace=false,         
    breaklines=true,                 
    captionpos=b,                    
    keepspaces=true,                 
    numbers=left,                    
    numbersep=5pt,                  
    showspaces=false,                
    showstringspaces=false,
    showtabs=false,                  
    tabsize=2,
    escapechar=|,
    frame=single
}

\lstset{style=mystyle}


\newcommand{\bi}{\begin{itemize}}
\newcommand{\li}{\item}
\newcommand{\ei}{\end{itemize}}
\newcommand{\Show}[1]{
\begin{center}
\shadowbox{\begin{minipage}{0.8\textwidth}
          #1
          \end{minipage}}
\end{center}
}
\newcommand{\arrow}{\ensuremath{\rightarrow}}

\newcommand{\uparr}{\ensuremath{\uparrow}}


\newcommand{\fig}[2]{\centerline{\includegraphics[width=#1\textwidth]{#2}}}

\newcommand{\bfr}[1]{\begin{frame}[fragile]\frametitle{{ #1 }}}
\newcommand{\efr}{\end{frame}}

\newcommand{\cola}{\begin{columns}\begin{column}{0.5\textwidth}}
\newcommand{\colb}{\end{column}\begin{column}{0.5\textwidth}}
\newcommand{\colc}{\end{column}\end{columns}}


\title{Parsing}
\author{Geoffrey Matthews}

\begin{document}

\begin{frame}
\maketitle
\end{frame}

\bfr{Modeling Syntax}
\bi
\li \lstinline{3 + 4} 
\li \lstinline{3 4 +}
\li \lstinline{(+ 3 4)}
\ei
\pause
\begin{lstlisting}
(define-type AE
  [num (n number?)]
  [add (lhs AE?) (rhs AE?)]
  [sub (lhs AE?) (rhs AE?)])
\end{lstlisting}
  

\begin{lstlisting}  
(add (num 3) (num 4))
\end{lstlisting}

\end{frame}

\bfr{Modeling Syntax}
\bi
\li \lstinline{(3 - 4) + 7} 
\li \lstinline{3 4 - 7 +}
\li \lstinline{(+ (- 3 4) 7)}
\ei
\begin{lstlisting}
(define-type AE
  [num (n number?)]
  [add (lhs AE?) (rhs AE?)]
  [sub (lhs AE?) (rhs AE?)])
\end{lstlisting}
  

\pause
\begin{lstlisting}  
(add (sub (num 3) (num 4))
     (num 7))
\end{lstlisting}
\end{frame}


\bfr{Read}
\begin{lstlisting}
> (read)
hello
'hello
> (read)
(+ 3 (- 4 5))
'(+ 3 (- 4 5))
> 'hello
'hello
> '(+ 3 (- 4 5))
'(+ 3 (- 4 5))
\end{lstlisting}

\end{frame}

\bfr{Read in Python}
\begin{lstlisting}
def parse(s):
    skip_blanks(s)
    if s[0] in string.digits:
        return parse_number(s)
    elif s[0] not in punctuation:
        return parse_symbol(s)
    elif s[0] == '(':
        s.pop(0)
        return parse_list(s)
\end{lstlisting}

\end{frame}

\bfr{Read in Python}
\begin{lstlisting}
def parse_number(s):
    skip_blanks(s)
    n = 0
    while s and s[0] in string.digits:
        n = n*10 + int(s.pop(0))
    return n

def parse_symbol(s):
    skip_blanks(s)
    w = ''
    while s and s[0] not in punctuation:
        w += s.pop(0)
    return w
\end{lstlisting}

\end{frame}

\bfr{Read in Python}
\begin{lstlisting}
def parse_list(s):
    skip_blanks(s)
    if s[0] == ')':
        s.pop(0)
        return []
    car = parse(s)
    cdr = parse_list(s)
    return [car] + cdr
\end{lstlisting}
\end{frame}

\bfr{Homework:  rewrite read in Scheme}

\bi
\li Translate into pure functional style:
\bi
\li No assignments statements.
\li No destructive operations (e.g. {\tt pop}).
\li No loops except recursion.
\ei
\li You will need to return two values:  
\bi
\li the parsed object
\li the remainder of the string
\ei
\li You need only handle the subset of Scheme handled
by our Python implementation.
\ei

\end{frame}

\bfr{Parsing Scheme into datatypes}

\begin{lstlisting}
;; parse : sexp -> AE
;; to convert s-expressions into AEs

(define (parse sexp)
  (cond [(number? sexp) (num sexp)]
        [(list? sexp)
         (case (first sexp)
           [(+) (add (parse (second sexp))
                     (parse (third sexp)))]
           [(-) (sub (parse (second sexp))
                     (parse (third sexp)))])]))
\end{lstlisting}

\begin{lstlisting}
> (parse '(+ 2 2))
(add (num 2) (num 2))
> (parse '(+ (- 2 3) 5))
(add (sub (num 2) (num 3)) (num 5))
> (parse '(+ (- 2 3) (- 12 3)))
(add
 (sub (num 2) (num 3))
 (sub (num 12) (num 3)))
\end{lstlisting}
\end{frame}

\bfr{Jobs for {\tt read} and {\tt parse}}
\bi
\li {\tt read} will reject the first,
\li {\tt parse}
will reject the second:
\begin{lstlisting}
  (+ 2 3 }
  (+ 2 3 4)
\end{lstlisting}
\ei

\end{frame}

\bfr{Interpreting {\em vs.} Interpreted Language}

\bi 
\li We will be using Scheme syntax for both languages.
\li Racket allows you to use any brackets: \verb|()[]{}|
\li We will use only braces \verb|{}| in the interpreted
languages.
\li Our interpreter will be written useing round and square
brackets \verb|()[]|
\li Scheme program:
\begin{verbatim}
    (+ (- 3 2) 5)
\end{verbatim}
\li Interpreted language program:
\begin{verbatim}
    {+ {- 3 2} 5}
\end{verbatim}
\ei
\end{frame}
\end{document}
