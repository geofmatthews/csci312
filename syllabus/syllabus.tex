\documentclass{article}
\usepackage{alltt}
\usepackage{color}
\usepackage{hyperref}
\setlength{\oddsidemargin}{-0.5in}
\setlength{\evensidemargin}{-0.5in}
\setlength{\textwidth}{7in}
\setlength{\textheight}{9in}
\setlength{\topmargin}{-0.5in}
\title{
Syllabus, CSCI312, Programming Languages
}
\author{Winter 2023}
\date{}

\newcommand{\myitem}[1]{\item[#1]}

\begin{document}
\maketitle

\begin{description}

\myitem{Instructor:}
Dr. Geoffrey Matthews,
Parmly 407A, {\tt gmatthews <at>  wlu <dot> edu}

\myitem{Web page:} \url{https://github.com/geofmatthews/csci312} 

\myitem{Office hours:} MWF 1:30-2:30

\myitem{Lectures:} MWF 12:15 - 1:15, Parmly 405

\myitem{Goals.}  This class is an introduction to the semantics of
programming languages.  We will study how things work: functions,
scoping, objects, continuations, and how these and other features
determine the character and style of various programming languages.

Rather than study semantics abstractly,
we will study these features using {\bf interpreter semantics}: 
to understand or explain how a feature works, build an interpreter
for it.  This method gives immediate feedback, allowing
us to program using this feature in simple languages,
and evaluate its usefulness  for various tasks.

Although we will do some simple programming in
a variety of programming languages, most of our programming will be in
Scheme, \url{http://www.schemers.org/}, an advanced functional
programming language.  Functional programming is a methodology, style,
and suite of programming language features that facilitates rapid
programming and algorithmic design.  Some required reading on the
usefulness of functional programming: {\em Beating the Averages},
\url{http://www.paulgraham.com/paulgraham/avg.html}.  Although we will
program in a single language, we will build interpreters for a wide
variety of languages.  Each language you encounter in your career will
partake of some of these features; understanding how they are
implemented, and what their runtime support entails, will give you a
deep understanding of any programming language you come across.


\myitem{Texts:}
\begin{itemize}
\item {\em Programming Languages, Application and Interpretation},
  Shriram Krishnamurthi,\\
\url{https://cs.brown.edu/~sk/Publications/Books/ProgLangs/2007-04-26/plai-2007-04-26.pdf}\\
This will be our textbook.  We will use the {\bf first} edition.

Note that the text is pedagogical; frequently he will start
with the obvious (but wrong) way to solve a problem, and then work his
way to the right solution 5 or 10 pages later.  If you don't spend the
time reading, and rereading, you may get confused.  However, if you
put in the time, you will understand at a much deeper level.


\item {\em Teach Yourself Scheme in Fixnum Days}, \\
\url{https://ds26gte.github.io/tyscheme/}\\
Best introductory Scheme tutorial I've found.  There are many
more online.

Note: skip Chapter 1, it deals with a different Scheme system.
Instead, run through the {\bf Quick} tutorial found
here: \url{https://docs.racket-lang.org/} or in
the Racket help desk available from DrRacket.

\item {\em The Scheme Programming Language}, (4/e),\\
\url{http://www.scheme.com/tspl4/}\\
Good reference manual for Scheme.
\end{itemize}

\myitem{Software:} 
\begin{itemize}
\item {\em The Racket programming language}, 
\url{https://racket-lang.org/}
\end{itemize}

\myitem{Grading:}
Homework: 50\%; Midterm: 20\%; Final: 30\%; Extra Credit: 10\%.

\item[Letter grades:]
A $\ge$ 90\% $>$ B $\ge$ 80\% $>$ C $\ge$ 70\% $>$ D $\ge$ 60\% $>$ F

\item[Homework:]  Homework assignments will be awarded as they
come up, to match timing with the lectures.  They will be announced
in class and on canvas.

Homework assignments will be available on the github site.
Several are there already, so you can look at them in advance,
but I reserve the right to revise them at any time prior to
their announcement in class and the setting of a due date.

Points will be awarded according to the
following.


\begin{tabular}{p{0.75\textwidth}|c}
\bf Factors to consider: & \bf Points \\\hline
{\bf Outstanding work.}    Well formatted, modular,
well commented, well designed.  
Clear, self-documenting identifiers: variables, functions, class names.
Good, consistent docstrings.
 Extra work on optional problems or extensions
of the required work.  Error checking.  Comprehensive unit tests. Innovative solutions.
Extensive documentation on design decisions and results.
 & 5 \\\hline
{\bf Good work.}  The problem is solved completely and without errors.
Adequate documentation.
 & 4 \\\hline
{\bf Adequate work.}  Most, but not all of the problem is solved.  Poor documentation.
 & 3 \\\hline
 {\bf Incomplete work.}  Some progress was made, but no complete solution.
 Nonexistent documentation.
 & 2 \\\hline
{\bf Poor work.}  Little or none of the problem is solved.  Random bits of code copied
from lectures or the problem description without showing any real coherence
or understanding of an approach to the problem.
 & 1 \\\hline
{\bf Unacceptable work.}
 Syntax errors.  Not turned in on time.  Did not follow instructions. & 0 \\
\end{tabular}
\item[Midterms:]  There will be two midterms, each worth 10\%, as
in the class schedule below.   They is open
book and open notes, but you may not consult with any classmates
or the internet or other resources.


\item[Final exam:]  The final exam is comprehensive.  It is open
book and open notes, but you may not consult with any classmates
or the internet or other resources.



\item[Extra credit:] A maximum of 10 percentage points of extra
credit may be given for a special project.  Special projects must: (a)
be proposed to the instructor, in writing, three weeks before the last
lecture, (b) be approved by the instructor at that time, (c) must
include a substantive software component, written in Scheme, (d) must
include a writeup with an introduction outlining the purpose of the
project, a section discussing the results, and a conclusion,
(e) must include well annotated source code, a user's guide, and a
programmer's guide to the software, and (f) must be complete and
turned in before the last lecture of class.

\newpage
\myitem{Schedule:}
\begin{alltt} \large
    January 2023
Su Mo Tu We Th Fr Sa
 8  9 10 11 12 13 14  Scheme intro
15 {\color{red}16} 17 18 19 20 21  Scheme intro, PLAI 1
22 23 24 25 26 27 28  PLAI 2,3; Python AST
29 30 31              PLAI 4
    February 2023
Su Mo Tu We Th Fr Sa
          1  2  3  4  PLAI 5,6
 5  6  7  8  9 10 11  PLAI 7,8,9
12 13 14 15 16 17 18  PLAI 10,11; review
19 {\color{red}20 21 22 23 24} 25  holiday
26 27 28              PLAI 12,13
     March 2023
Su Mo Tu We Th Fr Sa
          1  2  3  4  PLAI 13,14
 5  6  7  8  9 10 11  PLAI 15,16,17
12 13 14 15 16 17 18  PLAI 18,19,20
19 20 21 22 23 24 25  review
26 27 28 29 30 31     PLAI 24,25,26
     April 2023
Su Mo Tu We Th Fr Sa
                   1
 2  3  4  5  6  7  8  review
 9 {\color{red}10 11 12 13 14} 15  final exam
\end{alltt}


\end{description}
\end{document}
