\documentclass{beamer}
\usetheme{Copenhagen}
\usepackage{amsmath}
\usepackage{qtree}
%Information to be included in the title page:
\title{Simple Static Typechecking}
\author{Geoffrey Matthews}
\institute{Professor Emeritus}
\date{March 31, 2022}
\setbeamersize{text margin left=0.125in,text margin right=0.125in}
\begin{document}

\frame{\titlepage}

\begin{frame}
\frametitle{A Little Bit About Me}
\begin{itemize}
\item Born in California, grew up in the beachtown of Santa Barbara.
\pause
\item Ph.D. in the History and Philosophy of Science from Indiana University
\pause
\item Worked briefly in the defense industry for Hughes Aircraft
\pause
\item Professor of Computer Science at Western Washington University for 35 years
\pause
\item Research in Data Exploration, Visualization, Procedural Generation
\pause
\item Retired 2020 and moved to Virginia
\pause
\item Hablo Espa\~{n}ol
\pause
\item Eight cats
\end{itemize}
\end{frame}

\begin{frame}
\frametitle{Type Checking}
The idea is to {\em analyze} a program {\em statically} to find
errors that might occur when you run the program.  For example:
\begin{itemize}
\item Ensure that $\mathit{exp}_1$ and $\mathit{exp}_2$ are both type $\text{Int}$ in every case of
\begin{align*}
\mathit{exp}_1 + \mathit{exp}_2
\end{align*}
\item Ensure that $\mathit{exp}_1$ and $\mathit{exp}_2$ are both type $\text{Bool}$ in every case of
\begin{align*}
\mathit{exp}_1 \text{ Or } \mathit{exp}_2
\end{align*}
\item Ensure that $\mathit{exp}_1$ is type $\text{Bool}$ in every case of
\begin{align*}
\text{If }\mathit{exp}_1 \text{ Then } \mathit{exp}_2 \text{ Else } \mathit{exp}_3
\end{align*}
\end{itemize}
\end{frame}

\begin{frame}[fragile]
\frametitle{Recursive type checking}

\begin{verbatim}
(12 < 2) Or (true Or false)
\end{verbatim}
\Tree
  [.bool
    [.bool 
      [.int {12} ]
      {$<$}
      [.int {2} ]
      ]
    {\tt Or}
    [.bool
      [.bool {\tt true} ]
      {\tt Or}
      [.bool {\tt false} ]
      ]
      ]
      
\end{frame}

\begin{frame}[fragile]
\frametitle{Recursive type checking}

\begin{verbatim}
(12 + 2) Or (true Or false)
\end{verbatim}
\Tree
  [.fail
    [.int 
      [.int {12} ]
      {$+$}
      [.int {2} ]
      ]
    {\tt Or}
    [.bool
      [.bool {\tt true} ]
      {\tt Or}
      [.bool {\tt false} ]
      ]
      ]
      
\end{frame}

\begin{frame}[fragile]
\frametitle{Recursive type checking}

\begin{verbatim}
(12 + 2) - (x Or y)
\end{verbatim}
\Tree
  [.fail
    [.int 
      [.int {12} ]
      {$+$}
      [.int {2} ]
      ]
    {$-$}
    [.fail
      [.int {x} ]
      {\tt Or}
      [.int {y} ]
      ]
      ]
      
\end{frame}


\begin{frame}[fragile]
\frametitle{The language of Homework 5 has nothing to check!}
\begin{itemize}
\item The {\bf parser} ensures that only Bool expressions occur
in {\tt IF} and {\tt WHILE}
\item Only Int expressions can occur in assignments to variables
\item Only numbers, variables and arithmetic operators can occur
on the RHS of assignments
\item No program with a type error can even be parsed!
\item My original idea was to add boolean variables,\\ but I got a much better idea.
\end{itemize}

\end{frame}

\begin{frame}[fragile]
\frametitle{Quick review of {\tt Lambda} expressions}
My syntax:
\[
\underbrace{
  \text{\tt [ }
  \underbrace{\text{\tt \textbackslash x \{ x + 1\}}}_{\text{function}}
  \underbrace{\text{\tt (3 * 5)}}_{\text{argument}}
  \text{ \tt ]}
}_{\text{application}}
\]
\pause
\begin{itemize}
\item \verb|[ \x { x + 1 } (3 * 5) ]|  $\Rightarrow$ \pause \verb|  16| \pause
\item \verb|[ \f { [f 3] } \x { 2 * x } ]|  $\Rightarrow$ \pause \verb|  6|\pause
\item \verb|[ \x { \y { x + y} } 3 ]|  $\Rightarrow$ \pause \verb|\y { 3 + y }| \pause
\end{itemize}
\begin{itemize}
\item \verb|\x { x + 1 }|  :: \pause {\tt (int -> int)}\pause
\item \verb|\x { \y { x + y } }| :: \pause {\tt (int -> (int -> int))} \pause
\item \verb|\f { [f 3] }| :: \pause {\tt ((int -> ??) -> ??)}
\end{itemize}
\pause
To do static  recursive typechecking, we need 
{\bf typed lambda expressions}.

\end{frame}

\begin{frame}[fragile]
\frametitle{Typed {\tt Lambda} expressions}
My syntax:
\[
\underbrace{
  \text{\tt [ }
  \underbrace{\text{\tt \textbackslash x:int \{ x + 1\}}}_{\text{function}}
  \underbrace{\text{\tt (3 * 5)}}_{\text{argument}}
  \text{ \tt ]}
}_{\text{application}}
\]
\pause

\begin{itemize}
\item \verb|\f:(int -> int) { [f 3] }| :: \pause {\tt ((int -> int) -> int)}
\item \verb|\f:(int -> (int -> int)) { [f 3] }| :: \pause \\
{\tt ((int -> (int -> int)) -> (int -> int))}
\item \verb|\f:((int -> int) -> int) { [f x] }| :: \pause \\
{\tt (((int -> int) -> int) -> int)}
\end{itemize}

\end{frame}

\begin{frame}[fragile]
\frametitle{Typed {\tt lambda} expressions and applications in our language}

\begin{verbatim}
f := \x:int {x+1};  x := [f 3]   
         
g := [\x:int {\y:int {x+y}} 4];  x := [g 5]

j := \f:(int->int){\y:int{ [f y] };
k := [j \x:int { 2*x }];
x := [k 5]

x := [[\x:int { \y:int {x + y} } 13] 12]
y := [\x:int { [\y:int {x + y} 13] } 12]

\end{verbatim}
\pause
Are all of these type safe?

\pause
You will learn how to add {\tt lambda}s to your 
interpreters very soon.\\ Today we'll just do a typechecker for them.
\end{frame}

\begin{frame}[fragile]
\frametitle{Two new types of Arithmetic Expressions for Parsing}
\begin{verbatim}
data AExp =
   Var String
 | Num Int
 | Plus AExp AExp
 | Times AExp AExp 
 | Neg AExp
 | Div AExp AExp
 | Lambda String Type AExp  --NEW
 | App AExp AExp            --NEW
  deriving (Show, Eq)
\end{verbatim}

\vfill
\begin{itemize}
\item
We won't talk about the parsing today.
\item
{\tt AExp} are not always numbers anymore!
\end{itemize}
\end{frame}


\begin{frame}[fragile]
\frametitle{Example typechecking}
Expressions
\begin{verbatim}
[\x:int {99} 22] + 3                       --accept
[\x:(int->int) {99} 22]                    --reject
[\x:(int->int) {[x 2]} 9]                  --reject
[\x:(int->int) {[x 2]} \x:int{9}]          --accept
[\x:(int->int) {x + 2} \x:int{9}]          --reject
\end{verbatim}
Programs
\begin{verbatim}
x := \x:int{x+1};
IF x < 2 THEN y := 3 ELSE y := 4;          --reject

x := 3; 
z := [\y:int {x + y} 5]; 
w := x + y                                 --reject
\end{verbatim}
\end{frame}

\begin{frame}[fragile]
\frametitle{A type for types}
\begin{verbatim}
data Type =
     BoolType
   | IntType
   | LambdaType Type Type
   | FailureType 
    deriving (Show, Eq)

type TypeStore = Map.Map VarName Type



x := 5; y := \x:int {x+x};  z := \x:(int->int) { [x 4] }
\end{verbatim}

\end{frame}


\begin{frame}[fragile]
\frametitle{Typechecking Statements}
\begin{verbatim}
typeCheckStmt :: (Stmt AExp BExp) -> TypeStore 
                              -> (Bool, TypeStore)
typeCheckStmt program store =
 case program of
  ...
  Seq s1 s2 ->
    let (s1Good, store') = typeCheckStmt s1 store
    in if not s1Good
       then (False, store')
       else let (s2Good, store'')  = typeCheckStmt s2 store'
            in (s2Good, store'')
  ...


x := 5; y := \w:int {x+w};  z := [y x]
\end{verbatim}
\end{frame}

\begin{frame}[fragile]
\frametitle{Typechecking Statements}
\begin{verbatim}
typeCheckStmt :: (Stmt AExp BExp) -> TypeStore 
                              -> (Bool, TypeStore)
typeCheckStmt program store =
  case program of
    ...
    Skip -> (True, store)
    ...
\end{verbatim}
\end{frame}

\begin{frame}[fragile]
\frametitle{Typechecking Statements}
\begin{verbatim}
typeCheckStmt :: (Stmt AExp BExp) -> TypeStore 
                              -> (Bool, TypeStore)
typeCheckStmt program store =
 case program of
  ...    
  If b s1 s2 ->
    let (bType, store') = findTypeBExp b store in
    if bType /= BoolType
    then (False, store')
    else let (s1Good, store'') = typeCheckStmt s1 store' in
       if not s1Good
       then (False, store'')
       else let (s2Good, store''') = typeCheckStmt s1 store''
       in (s2Good, store''')
       -- Do you see a problem here?
\end{verbatim}
\pause
\begin{verbatim}
IF a < b THEN z := 3 ELSE z := \x:int{x+1} END; w := z + 1
\end{verbatim}
\end{frame}
\begin{frame}[fragile]
\frametitle{Typechecking Statements}
\begin{verbatim}
typeCheckStmt :: (Stmt AExp BExp) -> TypeStore 
                              -> (Bool, TypeStore)
typeCheckStmt program store =
 case program of
  ...
    While b s ->
      let (bType, store') = findTypeBExp b store in 
      if bType /= BoolType
      then (False, store')
      else let (sGood, store'') = typeCheckStmt s store' in
          (sGood, store'')   
  ...
\end{verbatim}
\end{frame}

\begin{frame}[fragile]
\frametitle{Typechecking Statements}
\begin{verbatim}
typeCheckStmt :: (Stmt AExp BExp) -> TypeStore 
                              -> (Bool, TypeStore)
typeCheckStmt program store =
 case program of
  ...
    Assign x val ->
      let (valType, store') = findTypeAExp val store
      in if valType /= FailureType
         then (True, Map.insert x valType store')
         else (False, store)
  ...
\end{verbatim}
\end{frame}

\begin{frame}[fragile]
\frametitle{Typechecking Boolean Expressions}
\begin{verbatim}
findTypeBExp :: (BExp AExp) -> TypeStore 
                     -> (Type, TypeStore)
findTypeBExp b store =
  case b of
    ...
    Bool x -> (BoolType, store)
    ...
\end{verbatim}
\end{frame}

\begin{frame}[fragile]
\frametitle{Typechecking Boolean Expressions}
\begin{verbatim}
findTypeBExp :: (BExp AExp) -> TypeStore 
                    -> (Type, TypeStore)
findTypeBExp b store =
  case b of
    ...
    Or x y -> 
    let (t, store') = (findTypeBExp x store) in
    if t /= BoolType
    then (FailureType, store')
    else let (t', store'') = (findTypeBExp y store') in
                if t' /= BoolType
                then (FailureType, store'')
                else (BoolType, store'')
    ...
\end{verbatim}
\end{frame}

\begin{frame}[fragile]
\frametitle{Typechecking Boolean Expressions}
\begin{verbatim}
findTypeBExp :: (BExp AExp) -> TypeStore 
                        -> (Type, TypeStore)
findTypeBExp b store =
  case b of
    ...
    Lt x y ->
     let (t, store') = (findTypeAExp x store) in
     if t /= IntType
     then (FailureType, store')
     else let (t', store'') = (findTypeAExp y store') in
                if t' /= IntType
                then (FailureType, store'')
                else (BoolType, store'')
    ...
\end{verbatim}
\end{frame}

\begin{frame}[fragile]
\frametitle{Typechecking Arithmetic Expressions}
\begin{verbatim}
findTypeAExp :: AExp -> TypeStore -> (Type, TypeStore)
findTypeAExp a store =
 case a of
  ...
  Num x -> (IntType, store)
  ...
\end{verbatim}

\vfill
\pause
Next: What if it's a variable?
\begin{verbatim}
data AExp =
  ...
  Var String
\end{verbatim}
\end{frame}

\begin{frame}[fragile]
\frametitle{Typechecking Arithmetic Expressions}
\begin{verbatim}
findTypeAExp :: AExp -> TypeStore -> (Type, TypeStore)
findTypeAExp a store =
 case a of
  ...
  Var x -> (Map.findWithDefault FailureType x store, store)
  ...
\end{verbatim}


\vfill
\pause

Next: What if it's a {\tt Plus}?
\begin{verbatim}
data AExp =
  ... Plus AExp AExp
\end{verbatim}
\end{frame}

\begin{frame}[fragile]
\frametitle{Typechecking Arithmetic Expressions}
\begin{verbatim}
findTypeAExp :: AExp -> TypeStore -> (Type, TypeStore)
findTypeAExp a store =
 case a of
  ...
    Plus x y -> 
    let (t, store') = (findTypeAExp x store) in
      if t /= IntType 
      then (FailureType, store')
      else let (t', store'') = (findTypeAExp y store') in
                     if t' /= IntType
                     then (FailureType, store'')
                     else (IntType, store'')
  ...
\end{verbatim}


\vfill
\pause

Next: What if it's a {\tt Lambda}?
\begin{verbatim}
data AExp = ... Lambda String Type AExp
\end{verbatim}
\end{frame}

\begin{frame}[fragile]
\frametitle{Typechecking Arithmetic Expressions}
\begin{verbatim}
findTypeAExp :: AExp -> TypeStore -> (Type, TypeStore)
findTypeAExp a store =
 case a of
  ...
    Lambda s t a -> 
    let (t', store') = 
       findTypeAExp a (Map.insert s t store) 
       in (LambdaType t t', store)                    
                     -- Why not store' ?
  ...
\end{verbatim}


\vfill
\pause

Next: What if it's a application of a function?
\begin{verbatim}
data AExp =
  ...
  App AExp AExp
\end{verbatim}
\end{frame}

\begin{frame}[fragile]
\frametitle{Typechecking Arithmetic Expressions}
\begin{verbatim}
findTypeAExp :: AExp -> TypeStore -> (Type, TypeStore)
findTypeAExp a store =
 case a of
  ...
    App f x -> 
    let (t,store') = (findTypeAExp f store) in
      case t of
        LambdaType t1 t2 -> 
           let (t',store'') = (findTypeAExp x store) in
           if t1 == t'
           then (t2, store)
           else (FailureType, store)
        _ (FailureType, store)
                 
                 -- what about store' or store''?
  ...
\end{verbatim}


\end{frame}
\begin{frame}[fragile]
\frametitle{Typed recursive functions}
What type information do you need to typecheck recursive functions?
\begin{verbatim}
REC f = \x:int { [ f (x+1) ] } IN [f 3] END

f :: (int -> ??)
\end{verbatim}

\pause
\begin{verbatim}
REC f = int \x:int { [ f (x+1) ] } IN [f 3] END
\end{verbatim}
\pause
\begin{verbatim}
f :: (int -> int)
\end{verbatim}
\pause
\begin{verbatim}
REC f = (int -> int) \x:int { [ f (x+1) ] } IN [f 3] END
\end{verbatim}
\pause
\begin{verbatim}
f :: (int -> (int -> int))
\end{verbatim}
\pause
Recursive typechecking will be one of your exercises today.
\end{frame}


\begin{frame}[fragile]
\frametitle{Your Turn! Write typecheckers for some new expressions.}

{\LARGE Let's Do {\tt IfExp} !}

\begin{verbatim}
IF x < 4 THEN 8 + x ELSE x + 9 END
\end{verbatim}

\bigskip
{\LARGE Let's Do {\tt Let} !}

\begin{verbatim}
LET x = 2 + 2 IN x + x END
\end{verbatim}

\bigskip
{\LARGE Let's Do Recursion!!}
\begin{verbatim}
REC f = int \x:int { [f x+1] } IN [f 3] END
\end{verbatim}

\bigskip


I've done the parsing for you, \\just add clauses  to the
{\tt findTypeAExp} function.

\end{frame}

\end{document}

