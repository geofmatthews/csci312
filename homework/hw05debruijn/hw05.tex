\documentclass[12pt]{article}
\pagestyle{empty}
\usepackage[margin=1in]{geometry}
\usepackage{multicol}

\usepackage{alltt}
\usepackage{tikz-qtree}
\usetikzlibrary{shadows,trees}
\usepackage{mathptmx}
\usepackage{graphicx}
%\newcommand{\includegraphics}{}



\newcommand{\lmb}{$\lambda$}
\newcommand{\bll}{\\\mbox{~~~~}$\bullet$}
\newcommand{\bk}[1]{$\langle${\bf #1}$\rangle$}

\newcommand{\lit}[1]{\mbox{\tt #1}}
\usepackage[listings]{tcolorbox}

\definecolor{codegreen}{rgb}{0,0.6,0}
\definecolor{codegray}{rgb}{0.5,0.5,0.5}
\definecolor{codepurple}{rgb}{0.58,0,0.82}
\definecolor{backcolour}{rgb}{0.95,0.95,0.92}

\lstdefinestyle{mystyle}{
    language=Lisp,
    backgroundcolor=\color{backcolour},   
    commentstyle=\color{codegreen},
    keywordstyle=\color{magenta},
    numberstyle=\tiny\color{codegray},
    stringstyle=\color{codepurple},
    basicstyle=\ttfamily\footnotesize,
    breakatwhitespace=false,         
    breaklines=true,                 
    captionpos=b,                    
    keepspaces=true,                 
    numbers=left,                    
    numbersep=5pt,                  
    showspaces=false,                
    showstringspaces=false,
    showtabs=false,                  
    tabsize=2,
    escapechar=|,
    frame=single
}

\lstset{style=mystyle}

\newcommand{\bi}{\begin{itemize}}
\newcommand{\li}{\item}
\newcommand{\ei}{\end{itemize}}

\author{CSCI 312 Homework 5}
\title{de Bruijn indices}

\begin{document}

\maketitle

\begin{description}

\item[de Bruijn indices:]  Write a procedure to transform WAE expressions
into equivalent expressions using de Bruijn indices, as in the examples
on page 25 of the text.  For example,
\begin{lstlisting}
{with {x 5}
    {with {y 3}
        {+ x y}}}
\end{lstlisting}
transforms into
\begin{lstlisting}
{with 5
    {with 3
        {+ <1> <0>}}}
\end{lstlisting}
And we convert
\begin{lstlisting}
{with {x 5}
    {with {y {+ x 3}}
        {+ x y}}}
\end{lstlisting}
into
\begin{lstlisting}
{with 5
    {with {+ <0> 3}
        {+ <1> <0>}}}
\end{lstlisting} 

\item[Data types:]
Use the text's definition of WAEs:
\begin{lstlisting}
(define-type WAE
  [num (n number?)]
  [add (left WAE?) (right WAE?)]
  [sub (left WAE?) (right WAE?)]
  [with (name symbol?) (named-exp WAE?) (body WAE?)]
  [id (name symbol?)])
\end{lstlisting}
Also use a datatype defining DBWAEs
\begin{lstlisting}
(define-type DBWAE
  [dbnum ...]
  [dbadd ...]
  [dbsub ...]
  [dbwith ...]
  [dbid ...])
\end{lstlisting}
where I will let you fill in the appropriate fields for each type.

\item[Examples:]
The two textbook examples will look like this:
\begin{lstlisting}
(db (with 'x (num 5)
          (with 'y (num 3)
                (add (id 'x) (id 'y)))) )
   =>

(dbwith
 (dbnum 5)
 (dbwith (dbnum 3) (dbadd (dbid 1) (dbid 0))))
\end{lstlisting}
 and
 \begin{lstlisting}
(db (with 'x (num 5)
          (with 'y (add (id 'x) (num 3))
                (add (id 'x) (id 'y)))) )
  =>
(dbwith
 (dbnum 5)
 (dbwith
  (dbadd (dbid 0) (dbnum 3))
  (dbadd (dbid 1) (dbid 0))))
\end{lstlisting}

\item[Optional 1:] Include a parser for WAEs and an ``unparser'' for DBWAEs,
so that the input/output can look like this:

\begin{lstlisting}
(debruijn
  `{with {x 5}
      {with {y {+ x 3}}
          {+ x y}}})
  =>
`(with 5
    (with (+ <0> 3)
        (+ <1> <0>)))
\end{lstlisting} 

\item[Optional 2:]  Include parsers and tranformers that will go the other
way, transforming DBWAEs into WAEs.


\item[Turn in:] Put all your files into a folder {\tt csci312hw05yourname},
zip it, and submit to canvas.


\end{description}

\end{document}

