\documentclass[12pt]{article}
\pagestyle{empty}
\usepackage[margin=1in]{geometry}
\usepackage{multicol}

\usepackage{alltt}
\usepackage{tikz-qtree}
\usetikzlibrary{shadows,trees}
\usepackage{mathptmx}
\usepackage{graphicx}
%\newcommand{\includegraphics}{}



\newcommand{\lmb}{$\lambda$}
\newcommand{\bll}{\\\mbox{~~~~}$\bullet$}
\newcommand{\bk}[1]{$\langle${\bf #1}$\rangle$}

\newcommand{\lit}[1]{\mbox{\tt #1}}
\usepackage[listings]{tcolorbox}

\definecolor{codegreen}{rgb}{0,0.6,0}
\definecolor{codegray}{rgb}{0.5,0.5,0.5}
\definecolor{codepurple}{rgb}{0.58,0,0.82}
\definecolor{backcolour}{rgb}{0.95,0.95,0.92}

\lstdefinestyle{mystyle}{
    language=Lisp,
    backgroundcolor=\color{backcolour},   
    commentstyle=\color{codegreen},
    keywordstyle=\color{magenta},
    numberstyle=\tiny\color{codegray},
    stringstyle=\color{codepurple},
    basicstyle=\ttfamily\footnotesize,
    breakatwhitespace=false,         
    breaklines=true,                 
    captionpos=b,                    
    keepspaces=true,                 
    numbers=left,                    
    numbersep=5pt,                  
    showspaces=false,                
    showstringspaces=false,
    showtabs=false,                  
    tabsize=2,
    escapechar=|,
    frame=single
}

\lstset{style=mystyle}

\newcommand{\bi}{\begin{itemize}}
\newcommand{\li}{\item}
\newcommand{\ei}{\end{itemize}}

\author{CSCI 312 Homework 7}
\title{Booleans in {\tt FAE}}

\begin{document}

\maketitle

\begin{description}

\item[Booleans in {\tt FAE}:]  Add booleans as a builtin type to the FAE
language (chapter 6).  Call this language {\tt BFAE}.  Thus, the expressible
values of the language now include numbers, closures, and booleans.
 
To make them useful you'll also need several other things:
\begin{itemize}
\item an {\tt if} statement
\item Boolean literals, for which we might as well use Scheme's literals,
\verb|#t| and \verb|#f|
\item Boolean functions:  {\tt and}, {\tt or}, and {\tt not}
\item Relational boolean functions:  \verb|<| and \verb|=|
\end{itemize}

Expand the {\tt FAE} language to handle all aspects of this
addition, includeing parsing and interpreting.

\item[Use it:]  Write some interesting programs in this language,
and test them to be sure they work.

\item[Turn in:] Put all your files into a folder {\tt csci312hw07yourname},
zip it, and submit to canvas.


\end{description}

\end{document}

