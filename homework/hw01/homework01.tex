\documentclass[12pt]{article}
\usepackage[margin=1in]{geometry}


%\usepackage{pstricks,pst-node,pst-tree}
\usepackage{amssymb,latexsym}
\usepackage{tikz}
\usepackage{graphicx}
\usepackage{fancyvrb}
\usepackage{hyperref}
\usepackage{fancybox}
\usepackage[listings]{tcolorbox}

\definecolor{codegreen}{rgb}{0,0.6,0}
\definecolor{codegray}{rgb}{0.5,0.5,0.5}
\definecolor{codepurple}{rgb}{0.58,0,0.82}
\definecolor{backcolour}{rgb}{0.95,0.95,0.92}

\lstdefinestyle{mystyle}{
    language=Python,
    backgroundcolor=\color{backcolour},   
    commentstyle=\color{codegreen},
    keywordstyle=\color{magenta},
    numberstyle=\tiny\color{codegray},
    stringstyle=\color{codepurple},
    basicstyle=\ttfamily\footnotesize,
    breakatwhitespace=false,         
    breaklines=true,                 
    captionpos=b,                    
    keepspaces=true,                 
    numbers=left,                    
    numbersep=5pt,                  
    showspaces=false,                
    showstringspaces=false,
    showtabs=false,                  
    tabsize=2,
    escapechar=|,
    frame=single
}

\lstset{style=mystyle}


\newcommand{\bi}{\begin{itemize}}
\newcommand{\li}{\item}
\newcommand{\ei}{\end{itemize}}
\newcommand{\Show}[1]{
\begin{center}
\shadowbox{\begin{minipage}{0.8\textwidth}
          #1
          \end{minipage}}
\end{center}
}
\newcommand{\arrow}{\ensuremath{\rightarrow}}

\newcommand{\uparr}{\ensuremath{\uparrow}}


\newcommand{\fig}[2]{\centerline{\includegraphics[width=#1\textwidth]{#2}}}


\author{CSCI 312 Homework 1}
\title{Working with Python Parse Trees}

\begin{document}

\maketitle
\begin{description}

\item[File names:]  Names of files and variables, when specified,
must be EXACTLY as specified.  This includes simple mistakes such
as capitalization.

\item[Individual work:]  All work must be your own.  Do not share
code with anyone other than the instructor and teaching assistants.
This includes looking over shoulders at screens with the code open.
You may discuss ideas, algorithms, approaches, {\em etc.} with
other students but NEVER actual code

\item[Project:] Use the Python AST module to make four
programs that handle arithmetic expressions, as follows:  
\begin{lstlisting}
>>> postfix('2 + (3 * 4) ** 5')
'2 3 4 * 5 ** +'
>>> infix('2 + (3 * 4) ** 5')
'(2+((3*4)**5))'
>>> prefix('2 + (3 * 4) ** 5')
'+ 2 ** * 3 4 5'
>>> calc('2 + (3 * 4) ** 5')
248834
\end{lstlisting}
The infix should be fully parenthesized, so spaces are not
needed.  The prefix and postfix will have exactly one
space between tokens and nothing else.

Also finish the unittest module provided on the website
with a more complete set of tests.

\item[Hand in:] Two files, \lstinline{ast_homework.py}
and \lstinline{test_ast_homework.py}. 

Place them in a
folder named {\tt csci312hw01<{\sl yourname}>} and zip
the folder.  Turn in the zipped folder on canvas before
the deadline.

You may use the procedures in \lstinline{ast_utilities.py}
but they are not necessary for the solution of this assignment.
If you use them, include the file in your folder.

Follow best practices for docstrings, for example,
as found here: \url{https://coderslegacy.com/python/best-practices-for-docstrings/}

The docstrings for the modules (files) should include
your name, the date, and the class name.


\end{description}


\end{document}
